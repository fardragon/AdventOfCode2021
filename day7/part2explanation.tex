\documentclass[a4paper,10pt]{article}
\usepackage[utf8]{inputenc}
\usepackage{amsmath}

%opening
\title{Advent of Code 2021 Day 7 Part 2}
\author{fardragon}

\begin{document}

\maketitle

Fuel usage for single ship from posistion $p_i$ to position $k$ :

\begin{equation}
\begin{split}
& f_s (p_i, k) = 1+2+ \dots + |k - p_i| \\
& = \sum^{|k - p+i|}_{n = 0} n \\
& = \frac{(k-p_i)^2 + |k-p_i|}{2}
\end{split}
\end{equation}


Fuel usage for all $s$ ships to position $k$ :
\begin{equation}
\begin{split}
& f(k) = \sum^{s}_{i=1} f_s(p_i, k) \\
& = \sum^{s}_{i=1} \frac{(k-p_i)^2 + |k-p_i|}{2}
\end{split}
\end{equation}

To find the solution we need to find the $k$ value that minimizes to function $f$.
To find a function minimum we need to calculate it's derivative:

\begin{equation}
f'(k) = \sum^{s}_{i=1} \frac{2(k-p_i) + sgn(k - p_i)}{2}
\end{equation}

If we comapre it to $0$ and rearrange:
\begin{equation}
\sum^{s}_{i=1} \frac{2(k-p_i) + sgn(k - p_i)}{2} = 0
\end{equation}

\begin{equation}
\sum^{s}_{i=1} p_i = sk + \sum^{s}_{i=1} \frac{sgn(k - p_i)}{2}
\end{equation}

This gives us and expression for k in almost closed form:

\begin{equation}
k = \frac{\sum^{s}_{i=1}p_i}{s} - \frac{1}{2}\frac{\sum^{s}_{i=1} sgn(k - p_i)}{s}
\end{equation}
\clearpage
The second term depends on k, but we may find it's upper and lower bounds. $\sum^{s}_{i=1} sgn(k - p_i)$ is at most $s$, if $p_i > k$ for all $i$. At minimum it is $-s$ , if $p_i < k$ for all $i$. Taking it into account we can bound the second term to values between $-\frac{1}{2}$ and $\frac{1}{2}$. Additionaly if we notice that the first term $\frac{\sum^{s}_{i=1}p_i}{s}$ is the arithmetic mean of $p_i$, the value $k$ is bounded by:
\begin{equation}
\overline{p} - \frac{1}{2} \leq k \leq \overline{p} + \frac{1}{2}
\end{equation}

\end{document}
